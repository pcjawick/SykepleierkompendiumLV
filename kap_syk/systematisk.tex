	\chapter{Hvordan jobbe systematisk}
		\section{Hva må du huske fra dette foredraget}
			\begin{itemize}
				\item Du må gjøre de du jobber sammen med så flinke som mulig, du spiller på et lag.\\
				\item Vi kommer alle til å gjøre feil. Meld fra sånn at de du jobber med slipper å gjenta feilen.\\
				\item Legen er også en del av teamet.\\
				\item Sjekklister hjelper oss å huske de kjedelige tingene vi vanligvis glemmer.\\
				\item Vær kritisk til all informasjon du får, du kan forvente at man skal kunne forklare hvorfor.\\
				\item Alle har ansvar for å lære, men det største ansvaret har du.\\
			\end{itemize}
		\section{Å lage verktøy for å ikke være redd}
			\paragraph{Hvorfor blir vi engstelige?\\}
				Det er utfordrende å jobbe i fremste rekke. Risikoen kan være høy. Men hva skjer dersom noe går galt. Mange av mottakerne av hjemmetjenester er eldre eller har flere sykdommer. Det kan være vanskelig kjenne igjen symptomer på alvorlige sykdommer, men også hva de forskjellige trenger. Noen ganger skjer ting som gjør helsepersonellet usikre eller redde.
			\paragraph{Glemsk?\\}
				Det er alltid mye nytt om behandling av kjente sykdommer. Det kan være veldig mye å sette seg inn i. Den menneskelige hjerne kan huske 7-9 ting på en gang. Det kan være utfordrende huske på alt sammen. Men samtidig kan vi som helsepersonell mye om sykdommer. Huskelister kan være en måte å redusere komplikasjoner på\cite{FA-gawande}.
		\section{Lær av feil}
			\paragraph{Endel av hverdagen vår\\}
				Å være helsepersonell vil gjøre at man må håndtere feil. Det er ingen som ønsker å gjøre feil, men det hender likevel alle. Når det skjer er det desto viktigere å lære av dem.
			\paragraph{I system\\}
				For å unngå å gjennta feil eller avdekke dem på systemnivå, må vi ha et system for å fange dem opp. Avviksmeldinger kan være kjedelig ekstraarbeid som kommer på toppen av alt i en travel hverdag. Likevel er det bare slik man kan lære.
		\section{Hvordan finne god informasjon}
			\paragraph{Jeg lurer på...\\}
				Hvor slår du opp hvis du lurer på noe. Det er ikke sikkert at alle internettkilder er like pålitelige. Hvem svarer på alle spørmålene som dukker opp i løpet av en travel arbeiddag. Det er veldig viktig å ha system på dette for å sørge for at all informasjonen vi bruker er kvalitetssikret. 
			\paragraph{Hvem bestemmer hvordan pasienter skal behandles?\\}
				Det finnes mange veiledere fra helsedirektoratet som gir retningslinjer. Andre ressurser er \href{http://www.helsebiblioteket.no/}{helsebiblioteket.no}. Det viktigste er at alle i tjenesten engasjeres i å utvikle faget, og at man får tid til det. 
		%\section{Sjekklister fra bakken og opp?}
		%\section{Systemer er kjedelig} Beholde?
		%\section{Noen praktiske prinsipper}