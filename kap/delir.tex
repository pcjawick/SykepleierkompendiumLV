\chapter{Delir}\label{delir}%Delir
		\section{Viktigeste momenter fra foredraget}
			\begin{itemize}
				\item Rammer nesten alltid de over 65 år.\\
				\item Delir er en brå forandring i det kognitive.\\
				\item Et symptom og ikke en sykdom.\\
				\item Har alltid en underliggende forverring eller nyoppstått sykdom, medisinbruk eller annen toksisk tilstand som årsak.\\
				\item Man behandler den utløsende årsaken. Haldol og andre medisiner er kun støttebehandling.\\
				\item Med urinprøve, et stetoskop og god sykehistorie kan man finne ut mye.\\
				\item Er ikke det samme som demens, men demente kan få det.\\
			\end{itemize}
		\section{Anatomi}
			Delir rammer hjernen og akkurat hva og hvordan den mentale forandringen foregår er ikke beskrevet i detalj\cite{rev_compre}\cite{pers_delir}.
		\section{Fysiologi}
			På grunn av det som er skrevet over er det vanskelig å gi en enkel forklaring på akkurat hva som fører til delir på cellenivå.
		\section{Patologi}
			Siden delir er symptom er det mangfoldige sykdommer som kan ligge bak. Her følger en kort oversikt\cite{legevakthandboka}.
				\paragraph{Oversikt over noen av tilstandene som går forut for delir\\}
					\begin{itemize}	
						\item Infeksjoner. Urinveisinfeksjon, pneumoni og sepsis, sjeldnere meningitt eller encefalitt.\\
						\item Medikamenter. Blant annet medikamenter med antikolinerg virkning, antiparkinsonmidler, opiater, sedativer, litium, digitoksin og blodtrykkssenkende midler.\\
						\item Metabolsk årsak. Blodsukkerendring. Tyreoideasykdom. Elektrolyttforstyrrelse, særlig hypo- og hypernatremi og hyperkalsemi. \\
						\item Syre- og baseforstyrrelse. Uremi (obs: nyresvikt, urinretensjon).\\
						\item Alkohol- eller rusmiddelseponering.\\
						\item Hypoksi, av ulike grunner, for eksempel hjertesvikt eller akutt redusert lungefunksjon.\\
						\item Kardiovaskulær årsak. Hjerteinfarkt, TIA, hjerneslag.\\
						\item Hodeskade. Obs — subduralt hematom.\\
						\item Frakturer. Obs — innkilt lårhalsbrudd hos demente.\\
						\item Epilepsi. Etter anfall.\\
						\item Underernæring. B-vitaminmangel.\\
						\item Hypo- og hypertermi.\\
					\end{itemize}

		\section{Klinikk}
			\paragraph{Mange symptomer}
				Utvikles vanligvis i løpet av timer til døgn. Oppmerksomhetssvikt, redusert hukommelse, desorientering. Gjerne døgnvariasjon i grad av forvirring. Uro og fikling (vandrer rundt, drar ut katetre eller venekanyler) eller tilbaketrukkethet. Noen hallusinerer. MMS skal ikke gjennomføres, fordi det gir ingen tilleggsinformasjon.
		\section{Pasienteksempler}
			\subsection{Pasient 9}
			\subsection{Pasient 10}